Se tiene determinada memoria compartida a la cual desean acceder dos tipos de procesos, lectores y escritores. Se deben cumplir las siguientes condiciones:
\begin{itemize}
	\item Solamente puede acceder un escritor al mismo tiempo. Si un lector o un escritor quiere ingresar a la memoria compartida y hay un escritor utilizándola, entonces debe esperar.
	\item Pueden haber varios lectores utilizando la memoria compartida al mismo tiempo, pero si un escritor desea hacer uso de la memoria mientras los lectores la usan, deberá esperar.
\end{itemize}

En otras palabras, la presencia de un $thread$ (lectores) en la memoria compartida no necesariamente excluye el ingreso de otros $threads$ pero, la presencia de determinado tipo de $thread$ (escritores), si lo hace.

Para resolver este problema implementamos dos estructuras de datos (RWLock\_libro y RWLock) que hacen uso de los semáforos POSIX. A continuación presentamos las dos soluciones:

\subsection{Implementación 1 (RWLock\_sem\_librito.h y RWLock\_sem\_librito.cpp)} 

En esta implementación utilizamos 3 semáforos y un contador inicializados de la siguiente manera:

\begin{algorithm}[H]
\caption{Inicialización}\label{ej1}
\begin{algorithmic}[1]
\Procedure{Inicialización}{}
	\State turnstile = Semaphore(1)
	\State mutex = Semaphore(1)
	\State roomEmpty = Semaphore(1)
	\State readers = 0
\EndProcedure
\end{algorithmic}
\end{algorithm}

En donde $mutex$ sirve para proteger la variable $readers$, $roomEmpty$ provee acceso exclusivo a la sección crítica y $turnstile$ evita la inanición de los escritores.

\begin{algorithm}[H]
\caption{Writers}\label{ej1}
\begin{algorithmic}[1]
\Procedure{Writers}{}
	\State turnstile.wait()
	\State roomEmpty.wait()
	\State //CRITICAL SECTION
	\State turnstile.signal()
	\State roomEmpty.signal()
\EndProcedure
\end{algorithmic}
\end{algorithm}

Si llega un escritor mientras hay lectores en la sección crítica entonces este se bloquea en la línea 3 esperando el lock de $roomEmpty$. Esto significa que obtuvo el lock de $turnstile$ que impedirá el acceso de más lectores mientras haya un escritor esperando (no hay inanición de escritores). A continuación, el código de los lectores:

\begin{algorithm}[H]
\caption{Readers}\label{ej1}
\begin{algorithmic}[1]
\Procedure{Readers}{}
	\State turnstile.wait()
	\State turnstile.signal()
	\State mutex.wait()
	\State readers++
	\If{readers==1}
		\State roomEmpty.wait()
	\EndIf
	\State mutex.signal()
	\State //CRITICAL SECTION
	\State mutex.wait()
	\State readers$--$
	\If{readers==0}
		\State roomEmpty.signal()
	\EndIf
	\State mutex.signal()
\EndProcedure
\end{algorithmic}
\end{algorithm}

El primer lector en ingresar espera que no haya nadie presente en la sección crítica (línea 6). Cuando el último lector se va, desbloquea $roomEmpty$ permitiendo que ingrese el escritor que estaba esperando. Cuando el escritor se va, desbloquea $turnstile$ permitiendo que cualquiera de los que estaba esperando (sea escritor o lector) acceda a la sección crítica. Esto último nos garantiza que la implementación se encuentra libre de inanición ya que tanto escritores como lectores tendrán acceso garantizado a la memoria compartida y no serán encolados por siempre.

\subsection{Implementación 2 (RWLock\_sem.h y RWLock\_sem.cpp)}

En esta implementación usamos 4 semáforos ($rw\_lock$, $accessR$, $accessW$ y $mutex$) y 4 contadores ($readers$, $readers\_waiting$, $writers$, $writers\_waiting$).
Incialización:

\begin{algorithm}[H]
\caption{Inicialización}\label{ej1}
\begin{algorithmic}[1]
\Procedure{Inicialización}{}
	\State rw\_lock = Semaphore(1)
	\State accessR = Semaphore(0)
	\State accessW = Semaphore(0)
	\State mutex = Semaphore(1)
	\State readers = 0
	\State readers\_waiting = 0
	\State writers = 0
	\State writers\_waiting = 0
\EndProcedure
\end{algorithmic}
\end{algorithm}

Código para los lectores:

\begin{algorithm}[H]
\caption{Readers}\label{ej1}
\begin{algorithmic}[1]
\Procedure{rlock}{}
	\State mutex.wait()
	\If{writers==0 \&\& writers\_waiting==0}
		\If{readers==0}
			\State rw\_lock.wait()
		\EndIf
		\State readers++
		\State mutex.signal()
	\Else
		\State readers\_waiting++
		\State mutex.signal()
		\State accessR.wait()
		\State mutex.wait()
		\If{readers==0}
			\State rw\_lock.wait()
		\EndIf
		\State readers++
		\State mutex.signal()
	\EndIf
	\State //CRTICAL SECTION
	
\EndProcedure
\end{algorithmic}
\end{algorithm}

Para que un lector pueda acceder a la sección crítica debe cumplirse que no haya escritores esperando ni escribiendo. Si se cumple, entonces se incrementa la variable $readers$ y se accede a la sección crítica. Si no ocurre, se incrementa $readers\_waiting$ y se espera el lock de $accessR$ que será desbloqueado cuando el último escritor salga. Si es el primer lector en llegar, se bloquea aguardando el lock de rw\_lock para cersiorarse de que no haya escritores escribiendo.

\begin{algorithm}[H]
\caption{Readers}\label{ej1}
\begin{algorithmic}[1]
\Procedure{runlock}{}
	\State mutex.wait()
	\State readers $--$
	\If{readers==0}
		\If{writers\_waiting > 0}
			\For{i $\shortleftarrow$ 0 to writers\_waiting}
				\State accessW.signal()
			\EndFor
			\State writers\_waiting $\shortleftarrow$ 0
			\State rw\_lock.signal()
		\EndIf
	\EndIf
	\State mutex.signal()
\EndProcedure
\end{algorithmic}
\end{algorithm}

Cuando el último lector sale verifica si hay escritores esperando. De ser así, tira tantos signals de $accessW$ como escritores haya. Luego reinicia la variable $writers\_waiting$ y desbloquea $rw\_lock$. Escritores:

\begin{algorithm}[H]
\caption{Writers}\label{ej1}
\begin{algorithmic}[1]
\Procedure{wlock}{}
	\State mutex.wait()
	\If{readers==0 \&\& readers\_waiting==0}
		\State writers++
		\State mutex.signal()
		\State rw\_lock.wait()
	\Else
		\State writers\_waiting++
		\State mutex.signal()
		\State accessW.wait()
		\State mutex.wait()
		\State writers++
		\State mutex.signal()
		\State rw\_lock.wait()
	\EndIf
\EndProcedure
\end{algorithmic}
\end{algorithm}

Los escritores realizan un procedimiento similar a los lectores con la diferencia de que solo puede acceder un escritor a la vez. Por lo tanto, siempre deben esperar el lock de $rw\_lock$.

\begin{algorithm}[H]
\caption{Writers}\label{ej1}
\begin{algorithmic}[1]
\Procedure{wunlock}{}
	\State mutex.wait()
	\State writers $--$
	\If{writers==0}
		\If{readers\_waiting > 0}
			\For{i $\shortleftarrow$ 0 to readers\_waiting}
				\State accessR.signal()
			\EndFor
			\State readers\_waiting $\shortleftarrow$ 0
		\EndIf
	\EndIf
	\State rw\_lock.signal()
	\State mutex.signal()
\EndProcedure
\end{algorithmic}
\end{algorithm}

Al igual que los lectores, el último escritor en salir despierta a todos los lectores que estaban esperando y reinicializa la variable $readers\_waiting$. Luego, desbloquea el $rw\_lock$.

\subsection{Tester}

Para corroborar el correcto funcionamiento del RWLock implementamos un tester (testerRW.cpp). Para utilizarlo es necesario pasarle dos parámetros: cantidad de lectores (L) y cantidad de escritores (E). El tester probará lo siguiente:

\begin{itemize}
	\item Que lean los L lectores y luego escriban los E escritores.
	\item Que escriban los E escritores y luego lean los L lectores.
	\item Que lean L/2 lectores y luego que escriban E/2 escritores. Después que lean los lectores restantes y finalmente los escritores.
\end{itemize}

La idea es observar que el algoritmo se encuentre libre de $deadlock$ por lo que al tomar instancias grandes de lectores y escritores el tester no debería tildarse.




