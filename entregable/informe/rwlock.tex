Se tiene determinada memoria compartida a la cual desean acceder dos tipos de procesos, lectores y escritores. Se deben cumplir las siguientes condiciones:
\begin{itemize}
	\item Solamente puede acceder un escritor al mismo tiempo. Si un lector o un escritor quiere ingresar a la memoria compartida y hay un escritor utilizándola, entonces debe esperar.
	\item Pueden haber varios lectores utilizando la memoria compartida al mismo tiempo, pero si un escritor desea hacer uso de la memoria mientras los lectores la usan, deberá esperar.
\end{itemize}

En otras palabras, la presencia de un $thread$ (lectores) en la memoria compartida no necesariamente excluye el ingreso de otros $threads$ pero, la presencia de determinado tipo de $thread$ (escritores), si lo hace.

Para resolver este problema implementamos dos estructuras de datos (RWLock\_libro y RWLock) que hacen uso de los semáforos POSIX. A continuación presentamos las dos soluciones:

\subsection{Implementación 1} 

En esta implementación utilizamos 3 semáforos y un contador inicializados de la siguiente manera:

\begin{algorithmic}
  \Function{Initialization}{}
	\State turnstile = Semaphore(1)
	\State mutex = Semaphore(1)
	\State roomEmpty = Semaphore(1)
	\State readers = 0
  \EndFunction
\end{algorithmic}

En donde $mutex$ sirve para proteger la variable $readers$, $roomEmpty$ provee acceso exclusivo a la sección crítica y $turnstile$ evita la inanición de los escritores.

\begin{algorithmic}
  \Function{Writers}{}
	\State turnstile.wait()
	\State roomEmpty.wait()
	\State //CRITICAL SECTION
	\State turnstile.signal()
	\State roomEmpty.signal()
  \EndFunction
\end{algorithmic}

Si llega un escritor mientras hay lectores en la sección crítica entonces este se bloquea esperando el lock de $roomEmpty$ lo que significa que obtuvo el lock de $turnstile$. Esto último impedirá el acceso de más lectores mientras haya un escritor esperando (no hay inanición de escritores). A continuación, el código de los lectores:

\begin{algorithmic}
  \Function{Readers}{}
	\State turnstile.wait()
	\State turnstile.signal()
	\State mutex.wait()
	\State readers++
	\If{readers==1}
		\State roomEmpty.wait()
	\EndIf
	\State mutex.signal()
	\State //CRITICAL SECTION
	\State mutex.wait()
	\State readers--
	\If{readers==0}
		\State roomEmpty.signal()
	\EndIf
	\State mutex.signal()
  \EndFunction
\end{algorithmic}

El primer lector en ingresar espera que no haya nadie presente en la sección crítica. Cuando el último lector se va, desbloquea $roomEmpty$ permitiendo que ingrese el escritor que estaba esperando. Cuando el escritor se va, desbloquea $turnstile$ permitiendo que cualquiera de los que estaba esperando (sea escritor o lector) acceda a la sección crítica. Esto último nos garantiza que la implementación se encuentra libre de inanición ya que tanto escritores como lectores tendrán acceso garantizado a la memoria compartida y no serán encolados por siempre.


