Implementar un RWLock sin inanici�n es important�simo para el buen desarrollo del juego aqu� presente, m�s aun si se considera que podr�a ser jugado por una cantidad muy grande de jugadores. Varios ataques podr�an no resultar como es esperado si hay ininici�n para los escritores.

Una implementaci�n de un m�todo de transici�n entre estados de lectura y escritura podr�a mejorar la protecci�n y desarrollo del juego, visto y considerando que podr�a por ejemplo permitir m�s concurrencia en la funci�n $ubicar$, donde por el simple hecho de que el �ltimo en ubicar sus barcos inicia el juego y modifica su estado no se permite la lectura en simult�neo de la variable $jugando$ para chequeo de errores. Lo mismo sucede en $tocar$ donde puede que finalice el juego.

Respecto al servidor, nuestra primera implementaci�n resolv�a las conexiones entrantes lanzando primero $n$ threads para cada jugador y el i-�simo thread realizaba el accept del i-�simo jugador. Resolvimos cambiar dicha implementacion para que un �nico thread se quede a la espera de nuevas conexiones y los nuevos threads de atenci�n se lanzan al momento de realizarse el $connect$ de cada uno. De esta manera no se tiene desde el principio $n$ threads $inactivos$ bloqueados. Algo similar sucede con los controladores, m�s importante es aun dado que no tenemos la certeza de cuantos controladores se conectar�n al juego.